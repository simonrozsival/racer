\chapter{Artificial Racing Agent}
\label{chapter:agent}

In this chapter we will first analyze how human racing drivers approach the task of car racing. We will then use this information to design the behavior of an autonomous agent and break the problem down into several smaller tasks. We will discuss how we can analyze the racing track and prepare it for trajectory planning.

\section{Racing Line}
\label{sec:racing_line}

When a human driver drives on a racing circuit, his or her task is to complete several laps around the circuit in the shortest possible time. In order to minimize the lap times, the driver trades off the total distance the car travels for the average speed at which it moves. The trajectory of the vehicle through the racing track is sometimes referred to as the racing line and it depends on the shape of the track and on the vehicle. The driver will follow different racing lines for different vehicles on the same circuit based on many of their different properties such as maximum speed, acceleration, braking power, or the grip of tires.

For a bend with a constant curvature there is a maximum speed at which a specific car can go and stay on the track. Exceeding this speed will cause a loss of friction between the tires and the road surface and the tires will not be able to exert enough lateral force to keep the car on the curve and the car will start to travel along a curve with a greater radius than the one intended by the driver. This situation is called understeer or oversteer, depending on whether it is the front or rear wheels which lose the friction and which wheels are powered by the engine.

\begin{figure}[t]
	\centering
	\label{fig:oversteer_understeer}
	\includegraphics[trim=250 180 20 0, clip,width=0.5\textwidth]{../img/understeer_oversteer.pdf}
	\caption{This figure shows the difference between oversteer and understeer. The red (leftmost) arrow shows oversteer, when the vehicle turns more tightly than it should and it collides with the inner boundary of the track. The purple (rightmost) arrow shows understeer as the vehicle does not turn enough and it collides with the outer boundary of the track of the corner.}
\end{figure}

On the other hand, for a constant vehicle speed, the vehicle can safely travel along any curve with a radius greater than the minimum safe radius. The driver will therefore try to follow curves with higher radii when going through corners but only to the point where the maximum safe speed reaches the maximum speed of the vehicle. Increasing of the curvature when it is not possible to increase the speed adds to the distance the car travels but it does not increase the average speed and therefore leads to a longer lap time.

The trajectory of the vehicle going through a corner can be split into several stages. First, the vehicle aligns itself with the outer edge of the track. Second, the vehicle must adjust its speed so it can safely stay on the curve. It is common to use the brakes mostly while the vehicle is moving straight to avoid locking the wheels in mid-turn. Third, the car starts turning at a turn-in point to go through the apex of the curve at the desired moment. Fourth, the vehicle goes through the apex, which is the point, where it is the closest to the inner edge of the track. Next, the vehicle starts exiting the corner and enters another part of the track.

\begin{figure}[b]
	\centering
	\includegraphics[trim=360 450 20 0, clip, width=0.75\textwidth]{../img/apex.pdf}
	\caption{Apex of the racing line is the point where the car is the closest to the inner boundary of the track. This figure shows two racing lines in a left turn. The red line is a classical racing where the car starts and ends at the track and ends at the outer boundary of the track. The late apex line is purple where the car starts braking and turning into the corner later and exits the corner at the inner boundary of the track.}
	\label{fig:apex}
\end{figure}

There are two common ways of going through a corner, a classical one and one called late apex. The classical way of leaving the corner is to keep the constant curvature of the turn and to align the vehicle with the outer edge of the track. The late apex is achieved by starting to turn into the corner later and exiting the corner much closer to the inner edge of the track than the classical way. During the late apex turn, the vehicle slows down more before entering the corner and it straightens the line before hitting the apex allowing for greater exit speed. The comparison between these two lines can be seen in Figure~\ref{fig:apex}.

Another factor which affects the speed at which the driver can go through a corner is the shape of the track around the corner. When there is a straight stretch following the corner, the driver can maximize the speed at which the vehicle leaves the corner to increase the average speed. If there is another corner immediately after the first one, the driver must plan ahead and adjust the speed before entering the first corner to a level at which the exit speed of the first corner will be appropriate to go through the following corner.

From the description of the racing behavior of human expert drivers, it is apparent that the key components of choosing the optimal racing line are the knowledge of the behavior of the vehicle and the shape of the track. The knowledge of the track gives the driver an opportunity to plan how he or she is going to approach driving on the track. The driver can plan how to approach each individual corner and what speeds and turning radii are optimal and still safe.

\section{Artificial Agent}

\begin{defn}\label{def:rational_agent}
    A rational agent is an entity which gathers information from the environment through sensors and changes the state of the environment in order to maximize a performance measure.
\end{defn}

A racing driver can be thought of as a rational agent as defined in Definition~\ref{def:rational_agent}. The driver observes the position of the vehicle on the racing track and the state of the vehicle as it moves along the track. He also observes the distance to the boundaries of the track and to any obstacles which can be present on the track. The driver reacts to these perceptions by giving control inputs to the vehicle through the steering wheel, brake and accelerator pedals, and shifting into different gears. The performance measure is the lack of collisions with the boundaries of the track or any obstacles on the track and minimizing of the lap time.

An artificial racing agent would use electronic sensors such a LIDAR, wheel encoders, an \gls{IMU}, or a camera to observe the state of the environment. Based on this observation, the agent can estimate its state in the world and the state of other entities in the world, such as obstacles or opponents. Based on this information, the agent has to select a control input for the actuators of the vehicle.

The agent can use the information about its initial state and calculate a time-optimal racing line from this initial state through the whole circuit up to the finish line using a trajectory planning algorithm. Depending on the size of the circuit, this could be a computationally expensive operation. If later on there is a need to re-plan the trajectory during the race, because the vehicle could not follow the trajectory accurately and it is necessary to come up with a contingency plan, the vehicle might have to repeat the expensive calculation. Instead, the agent can analyze the shape of the circuit and identify the corners and focus its effort only on the next two or three corners ahead of him.

\subsection{Decision Making Process}

An agent is sometimes described in the form of an agent function. This function takes the data from the sensors and outputs a command to the actuators. The command is then executed and it has an effect on the environment. In the next step we measure the changed state of the environment and the agent reacts to this changed state. The agent can select the next command in an order to correct the outcome of the previous command when it is different from the predicted outcome. This process is then repeated over and over in a so-called closed-feedback loop.

In order to avoid describing the logic of the racing agent in a single complicated function, we can divide the decision making process of the agent into several smaller independent sub-problems: localization, track analysis and waypoint selection, trajectory planning, and trajectory following. Some of these sub-problems output information which is necessary as an input for another of these sub-problems they form nodes of a dependency graph of the decision making process which is visualized in Figure~\ref{fig:racing_agent_diagram}.

There is one more reason to break the task down into several sub-problems. The rate at which the individual sub-problems produce outputs is different and they work asynchronously. While the trajectory following algorithm should react to any location update with an action to correct the movement of the vehicle and it should do this many times per second, the planning process of a trajectory will take some non-trivial time and so the trajectory planning process will have a much lower output frequency.

\begin{figure}[b]\centering
	\includegraphics[width=125mm]{../img/racing_agent_diagram.pdf}
	\caption{A diagram of the decision making process of the racing agent.}
	\label{fig:racing_agent_diagram}
\end{figure}

\paragraph{Current state} As the vehicle moves on the track, the agent needs to know its current state: its position on the map, orientation, and speed. A localization algorithm collects the data from different sensors and estimates the current state of the vehicle. The accuracy and the frequency of state updates depend greatly on the capabilities of the sensors and the processing power of the on-board computer. The agent takes the raw data from a localization algorithm and produces a convenient representation of the current state for track monitoring, trajectory planning, and trajectory following.

\paragraph{Track analysis \& monitoring} Before the race begins, the agent analyses the track and finds the corners and bends on the track and marks a coordinate of a point near the apex of the corner and forms a list of waypoints along the track. During the race, the agent will keep track of which of the waypoints discovered during track analysis it drove past the last. The waypoint selection process will publish the following $n$ waypoints as the next goal for trajectory planning. The parameter $n\in\mathbb{N}$ is the lookahead of the agent. The selection of this parameter is a trade off between the quality of the trajectory and the size of the configuration space that will be searched. This directly affects the update rate of the trajectory planning process.

\paragraph{Trajectory planning} The trajectory planning algorithm finds a feasible trajectory from the last state of the vehicle estimated by the localization algorithm through the selected waypoints. The trajectory planning process will start planning a new trajectory as soon as it finishes planning the previous one. As the agent drives along the circuit and drives past a waypoint, the trajectory planning algorithm receives a new sequence of waypoints and the next trajectory it plans will account also for the next corner of the circuit within the lookahead instance. Planning can be a slow process even when we try to reduce the size of the search space as much as possible. For the practical application, the planning algorithm must update the trajectories faster than the vehicle moves through the circuit. If the agent comes to the end of an old reference trajectory and it has no update, it has to stop and wait for an update. This would obviously affect the lap time which is undesirable.

\paragraph{Trajectory following} The latest reference trajectory and the current estimate of the state of the vehicle is used to calculate the next command for the actuators of the vehicle. In the ideal case, this trajectory following algorithm will send the commands to the hardware at the maximum possible rate of which the hardware is capable. In practice we are limited by the rate at which we receive the state estimates from the localization algorithm. The trajectory following algorithm is also responsible for collision avoidance either by actively guiding the vehicle around the obstacles it might hit, or by slowing down the vehicle while waiting for a new reference trajectory.

